\documentclass{beamer}
\mode<presentation>
{
\usetheme{Warsaw}
\setbeamercovered{transparent}
}

\usepackage[english]{babel}			% Slides
\usepackage[latin1]{inputenc}		% Slides
\usepackage{mathptmx}				% Font
\usepackage[scaled=.90]{helvet}		% Font
\usepackage{courier}				% Font
\usepackage[T1]{fontenc}			% Font
\usepackage{soul}					% Font (underline thickness)
\usepackage{multirow}				% Tables
\usepackage{adjustbox}				% Tables
\usepackage{amsmath}				% Math
\usepackage{amsfonts}				% Math
\usepackage{amsthm}					% Math
\usepackage{amssymb}				% Math
\usepackage{pzccal}
\usepackage{verbatim}
\usepackage{alltt}



%%%%%%%%%%%%%
%% OPTIONS %%
%%%%%%%%%%%%%
\setbeamertemplate{caption}[numbered]
\graphicspath{{./figures/}}
\newcommand{\indep}{\rotatebox[origin=c]{90}{$\models$}}
\newcommand{\E}{\mathbb{E}}


%%%%%%%%%%%%%%%%%
%% INFORMATION %%
%%%%%%%%%%%%%%%%%
\title{Reproducibility}
\subtitle{Recreating results using R-packages}
\author{Linh Tran, MPH}
\date{March 10, 2015}
\institute[Universities of]
{
Group in Biostatistics \\
Univ of California, Berkeley
}


\begin{document}

%%%%%%%%%%%
%% TITLE %%
%%%%%%%%%%%
\begin{frame}
\titlepage
\pgfdeclareimage[height=0.75cm]{university-logo}{./figures/Berkeley_Logo}
\pgfdeclareimage[height=0.75cm]{department-logo}{./figures/School_Logo}
  \makebox[0.95\paperwidth]{%
	\pgfuseimage{department-logo}~
	\hspace{3in}
	\pgfuseimage{university-logo}~
	\hspace{0.45in}
  }%
\end{frame}


%%%%%%%%%%%%%%%%%%%%%%%%%%%%%%%%%%%%%%%%%%%%%%%%%%%%%%%%%%%%%%%%%%%%%%
%%%%%%%%%%%%%%%%%%%%%%%%%%%% INTRODUCTION %%%%%%%%%%%%%%%%%%%%%%%%%%%% 
%%%%%%%%%%%%%%%%%%%%%%%%%%%%%%%%%%%%%%%%%%%%%%%%%%%%%%%%%%%%%%%%%%%%%%
\section{Introduction}
\begin{frame}
\fontsize{18}{7.2}\selectfont
\centering
\textbf{INTRODUCTION}
\end{frame}


%%%%%%%%%%%%%%%%
%% BACKGROUND %%
%%%%%%%%%%%%%%%%
\subsection{Background}
\begin{frame}
\frametitle{Background}
Fields requiring reproducibility:
\begin{itemize}
  \item Biology
  \item Chemistry
  \item Physics
\end{itemize}
\begin{center}
  \textbf{We're no exception!} \\
\end{center}
\uncover<2->{
In fact, our field benefits from \ul{exact} reproducibility
\begin{itemize}
  \item Given same data and code, we should always get the same answers.
  \item We should embrace this!
\end{itemize}
}
\end{frame}


%%%%%%%%%%%%%%%%%%%%%%%%%%
%% REPRODUCE APPROACHES %%
%%%%%%%%%%%%%%%%%%%%%%%%%%
\begin{frame}
\frametitle{Background}
Ways to reproduce:
\begin{enumerate}
  \item Series of R scripts
  \item The \texttt{ProjectTemplate} R package
  \item Creating your own R-package
  \begin{itemize}
  	\item What we're focusing on today
	\item Using S3 object system (cf. S4 system)
  \end{itemize}
\end{enumerate}
\end{frame}



%%%%%%%%%%%%%%%%%%%%%%%%%%%%%%%%%%%%%%%%%%%%%%%%%%%%%%%%%%%%%%%%%%%%%%
%%%%%%%%%%%%%%%%%%%%%%%%%%%%% R-PACKAGSE %%%%%%%%%%%%%%%%%%%%%%%%%%%%% 
%%%%%%%%%%%%%%%%%%%%%%%%%%%%%%%%%%%%%%%%%%%%%%%%%%%%%%%%%%%%%%%%%%%%%%
\section{R packages}
\begin{frame}
\fontsize{18}{7.2}\selectfont
\centering
\textbf{R-PACKAGES}
\end{frame}


%%%%%%%%%%%%%%%%
%% R-PACKAGES %%
%%%%%%%%%%%%%%%%
\subsection{Introduction}
\begin{frame}
\frametitle{R-packages}
\textbf{R-package:} a collection of R code and documentation. \\
\vspace{.05in}
Files/directories needed:
\begin{itemize}
  \item DESCRIPTION: File describing package
  \item "R" subfolder: Stores R scripts
\end{itemize}
Files/directories possibly desired:
\begin{itemize}
  \item NAMESPACE: File specifying public/private objects
  \item "data" subfolder: Stores datasets used by package
  \item "man" subfolder: Stores documentation files explaining code/data in package
  \item "inst" subfolder: Files/folders to be copied at build time
  \item "src" subfolder: C/Fortran source code (if used)
  \item "test" subfolder: Tests for R code
  \item "vignettes" subfolder: Further author customized documentation
\end{itemize}
Built packages can be stored on a CRAN mirror as source or binary (Windows/Mac).
\end{frame}


%%%%%%%%%%%%%%%%%
%% DESCRIPTION %%
%%%%%%%%%%%%%%%%%
\begin{frame}[fragile]
\frametitle{DESCRIPTION}
Specified in so called Debian-control-file format (i.e. Keyword: Value). e.g.
\begin{verbatim}
Package: randomForest
Title: Breiman & Cutler's random forests 
Version: 4.6-10
Date: 2014-07-17
Depends: R (>= 2.5.0), stats
Suggests: RColorBrewer, MASS
Author: Fortran original by Leo Breiman and Adele Cutler, R port by
        Andy Liaw and Matthew Wiener.
etc...
\end{verbatim}
List of all fields can be found at cran.r-project.org under \it{"Writing R Extensions"}
\end{frame}


%%%%%%%%%%%%%%%
%% NAMESPACE %%
%%%%%%%%%%%%%%%
\begin{frame}[fragile]
\frametitle{NAMESPACE}
Specifies R code/packages/objects/classes/methods to be imported or exported. e.g.
\begin{verbatim}
import(Biobase)
useDynLib(randomForest)
export(combine, getTree, grow, importance, margin, MDSplot, na.roughfix,
       partialPlot, randomForest, rfImpute, treesize, tuneRF,
       varImpPlot, varUsed, rfNews, outlier, classCenter, rfcv)
S3method(print, randomForest)
S3method(predict, randomForest)
S3method(plot, randomForest)
etc...
\end{verbatim}
Sealed once package is installed, though hidden function can still be called with "\texttt{package:::function}"
\end{frame}


%%%%%%%%%%%%%%%%%%%
%% DOCUMENTATION %%
%%%%%%%%%%%%%%%%%%%
\begin{frame}[fragile]
\frametitle{Package documentation}
Files stored in "man" subfolder and displayed anytime a user calls \texttt{help("function")}. Format similar to Latex, e.g. for \texttt{grow.RD} from \texttt{randomForest}:
\begin{verbatim}
\name{grow}
\alias{grow}
\alias{grow.default}
\alias{grow.randomForest}
\title{Add trees to an ensemble}
\description{
  Add additional trees to an existing ensemble of trees.
}
etc...
\end{verbatim}
List of all fields can be found at cran.r-project.org under \it{"Writing R Extensions"}
\end{frame}


%%%%%%%%%%
%% DATA %%
%%%%%%%%%%
\begin{frame}
\frametitle{data directory}
Stores data sets that can be loaded with package
\begin{itemize}
  \item Stored as ".rda" or ".RData" files
  \item Can be called with \texttt{data("name")} once package is loaded
  \item Documentation files can also be created for these saved data sets
  \begin{itemize}
  	\item Details for documenting can be found at cran.r-project.org under \it{"Writing R Extensions"}
  \end{itemize}
\end{itemize}
\end{frame}


%%%%%%%%%%%%%%%%%%%%%%%
%% CREATING PACKAGES %%
%%%%%%%%%%%%%%%%%%%%%%%
\subsection{Creating R packages}
\begin{frame}
\frametitle{Creating packages}
Now that we understand what a package is, how do we create them?
\begin{enumerate}
  \item<2-> Manually creating each file / subfolder
  \item<3-> Using the \texttt{package.skeleton()} function
  \begin{itemize}
  	\item<3-> Sets up directory / file templates for us
  \end{itemize}  
  \item<4-> Using the \texttt{devtools} R-package
  \begin{itemize}
  	\item<4-> Sets up directory / file templates for us
  	\item<4-> Developed by Hadley Wickham to facilitate package creation
	\item<4-> Well documented from his website: http://r-pkgs.had.co.nz
	\item<4-> My preferred approach
  \end{itemize}
\end{enumerate}
\end{frame}



%%%%%%%%%%%%%%%%%%%%%%%%%%%%%%%%%%%%%%%%%%%%%%%%%%%%%%%%%%%%%%%%%%%%%%
%%%%%%%%%%%%%%%%%%%%%%%%%%%%%% DEVTOOLS %%%%%%%%%%%%%%%%%%%%%%%%%%%%%% 
%%%%%%%%%%%%%%%%%%%%%%%%%%%%%%%%%%%%%%%%%%%%%%%%%%%%%%%%%%%%%%%%%%%%%%
\section{devtools}
\begin{frame}
\fontsize{18}{7.2}\selectfont
\centering
\textbf{DEVTOOLS}
\end{frame}


%%%%%%%%%%%%%%
%% DEVTOOLS %%
%%%%%%%%%%%%%%
\begin{frame}
\frametitle{devtools}
Using devtools allows us many functions which streamline different stages of creating an R package. e.g.
\begin{itemize}
  \item \texttt{create("path/to/package/pkgname")}
  \begin{itemize}
  	\item[$-$] Creates parent directory for package
  \end{itemize}
  \item \texttt{load\_all("path/to/package/pkgname")}
  \begin{itemize}
  	\item[$-$] Loads package without having to build
  \end{itemize}
  \item \texttt{document("path/to/package/pkgname")}
  \begin{itemize}
  	\item[$-$] Uses Roxygen to document R code / data sets
  \end{itemize}
  \item \texttt{use\_vignette("filename")}
  \begin{itemize}
  	\item[$-$] Creates vignette directory and file
  \end{itemize}
\end{itemize}
\textbf{Let's see it in action!}
\end{frame}


%%%%%%%%%%%%%%%%
%% SIMULATION %%
%%%%%%%%%%%%%%%%
\begin{frame}
\frametitle{Simulation}
Let's do a quick simulation to show that the estimator solving the efficient influence function has lower variance. 
\vspace{.1in}
Given
\begin{itemize}
  \item $O=(W,A,Y) \sim P_0 \in \mathcal{M}$
  \item $P_0(O) = P_0(W)P_0(A|W)P_0(Y|A,W)$, where
  \begin{itemize}
  	\item $W \sim N(0,.75^2)$
	\item $A|W \sim Ber(logit^{-1}(-0.5 - 1.5*W))$
	\item $Y|A,W \sim Ber(logit^{-1}(-0.75 + 1.2*W - 0.75*A))$
  \end{itemize}
\end{itemize}
Our target parameter is
\begin{equation}
	\Psi(P_0) = \mathbb{E}_{W}[\mathbb{E}_{P_0}[Y|A=1,W]] \approx 0.2149 
\end{equation}
\end{frame}


%%%%%%%%%%%%%%%%
%% ESTIMATORS %%
%%%%%%%%%%%%%%%%
\begin{frame}
\frametitle{Estimators}
Estimators we'll be evaluating
\begin{itemize}
  \item g-computation
  \begin{gather*}
  	\hat{\psi}_n^{gcomp} = \mathbb{E}_n \bar{Q}_n(Y|A=1,W)
  \end{gather*}
  \item Inverse Probability Treatment Weights (IPTW)
  \begin{gather*}
  	\hat{\psi}_n^{IPTW} = \mathbb{E}_n \left[ Y \frac{\mathbb{I}(A=1)}{g_n(A|W)} \right]
  \end{gather*}
  \item Augmented-IPTW (A-IPTW)
  \begin{gather*}
  	\hat{\psi}_n^{AIPTW} = \mathbb{E}_n \left[ (Y - \bar{Q}_n(Y|A=1,W))\frac{\mathbb{I}(A=1)}{g_n(A|W)} \right] + \bar{Q}_n(Y|A=1,W)
  \end{gather*}
\end{itemize}

nb. It would be very beneficial to work out the Efficient Influence Curve on your own.
\end{frame}


%%%%%%%%%%%%%%%%%%%%%%%%%%%%%%%%%%%%%%%%%%%%%%%%%%%%%%%%%%%%%%%%%%%%%%
%%%%%%%%%%%%%%%%%%%%%%%%%%%%%% VIGNETTES %%%%%%%%%%%%%%%%%%%%%%%%%%%%%
%%%%%%%%%%%%%%%%%%%%%%%%%%%%%%%%%%%%%%%%%%%%%%%%%%%%%%%%%%%%%%%%%%%%%%
\section{Vignettes}
\begin{frame}
\fontsize{18}{7.2}\selectfont
\centering
\textbf{VIGNETTES}
\end{frame}


%%%%%%%%%%%%%%%
%% VIGNETTES %%
%%%%%%%%%%%%%%%
\begin{frame}
\frametitle{Vignettes}
A way of storing author customized documentation to support package
\begin{itemize}
  \item Can be stored in multiple (pre-compiled) formats including
  \begin{itemize}
  	\item R markdown
	\item Sweave
	\item knitr
  \end{itemize}  
  \item Currently compiled to pdf at build time
  \item Can be called from R console
\end{itemize}
\bf Let's look at examples and make one.
\end{frame}


\end{document}



